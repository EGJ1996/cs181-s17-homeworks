
\documentclass[submit]{harvardml}

% Put in your full name and email address.
\name{Your Name}
\email{email@fas.harvard.edu}

% List any people you worked with.
\collaborators{%
  John Doe,
  Fred Doe
}

% You don't need to change these.
\course{CS181-S17}
\assignment{Assignment \#3}
\duedate{5:00pm March 24, 2016}

\usepackage[OT1]{fontenc}
\usepackage[colorlinks,citecolor=blue,urlcolor=blue]{hyperref}
\usepackage[pdftex]{graphicx}
\usepackage{subfig}
\usepackage{fullpage}
\usepackage{amsmath}
\usepackage{amssymb}
\usepackage{color}
\usepackage{todonotes}
\usepackage{listings}
\usepackage{common}
\usepackage{bm}

\usepackage[mmddyyyy,hhmmss]{datetime}

\definecolor{verbgray}{gray}{0.9}

\lstnewenvironment{csv}{%
  \lstset{backgroundcolor=\color{verbgray},
  frame=single,
  framerule=0pt,
  basicstyle=\ttfamily,
  columns=fullflexible}}{}

\begin{document}
\begin{center}
{\Large Homework 3: Max-Margin and SVM}\\
\end{center}
\subsection*{Introduction}

This homework assignment will have you work with max-margin methods
and SVM classification. The aim of the assignment is (1) to further
develop your geometrical intuition behind margin-based classification
and decision boundaries, (2) to explore the properties of kernels and
how they provide a different form of feature development from
basis functions, and finally (3) to implement a basic Kernel based
classifier.

There is a mathematical component and a programming component to this
homework.  Please submit your PDF and Python files to Canvas, and push
all of your work to your GitHub repository. If a question requires you
to make any plots, like Problem 3, please include those in the
writeup.

\newpage
%%%%%%%%%%%%%%%%%%%%%%%%%%%%%%%%%%%%%%%%%%%%%
% Problem 1
%%%%%%%%%%%%%%%%%%%%%%%%%%%%%%%%%%%%%%%%%%%%%
\begin{problem}[Fitting an SVM by hand, 7pts]
  For this problem you will solve an SVM without the help of a
  computer, relying instead on principled rules and properties of
  these classifiers.

Consider a dataset with the following 7 data points each with $x \in \reals$ : \[\{(x_i, y_i)\}_i =\{(-3
, +1 ), (-2 , +1 ) , (-1,  -1 ), (0, -1), ( 1 , -1 ), ( 2 , +1 ), ( 3 , +1 )\}\] Consider
mapping these points to $2$ dimensions using the feature vector $\bphi(x) =  (x, x^2)$. The hard margin classifier training problem is:
%
\begin{align}
  &\min_{\mathbf{w}, w_0} \|\mathbf{w}\|_2^2 \label{eq:dcp} \\
  \quad \text{s.t.} \quad & y_i(\mathbf{w}^\top \bphi(x_i) + w_0) \geq 1,~\forall i \in \{1,\ldots, n\}\notag
\end{align}

The exercise has been broken down into a series of questions, each
providing a part of the solution. Make sure to follow the logical structure of
the exercise when composing your answer and to justify each step.

\begin{enumerate}
\item Plot the training data in $\reals^2$ and draw the decision boundary
of the max margin classifer.
%
\item  What is the value of the margin achieved by the optimal
decision boundary? 
%
\item What is a vector that is orthogonal to the decision boundary?

%
\item Considering discriminant $h(\bphi(x);\boldw,w_0)=\boldw^\top\bphi(x) +w_0$, 
give an expression for {\em all possible} $(\boldw,w_0)$ that define
the optimal decision boundary. Justify your answer.

%
%dcp: answer is $a x^2-b=0$, $a>0$, $b/a=5/2$.
%


  \item Consider now the training problem~\eqref{eq:dcp}. Using
your answers so far, what particular solution
to $\boldw$ will be optimal for this optimization
problem?
%%
%

  \item Now solve for
the corresponding value of $w_0$, using your general expression 
from part~(4.) for the optimal decision boundary.
Write down the discriminant
function $h(\bphi(x);\boldw,w_0)$.


\item What are the support
vectors of the classifier?
 Confirm that the solution in part~(6.) makes the constraints in~\eqref{eq:dcp} binding
for support vectors.

\end{enumerate}

\end{problem}
\subsection*{Solution}




\newpage
%%%%%%%%%%%%%%%%%%%%%%%%%%%%%%%%%%%%%%%%%%%%%
% Problem 2
%%%%%%%%%%%%%%%%%%%%%%%%%%%%%%%%%%%%%%%%%%%%%
\begin{problem}[Composing Kernel Functions, 10pts]


  A key benefit of SVM training is the ability to use kernel functions
  $K(\boldx, \boldx')$ as opposed to explicit basis functions
  $\bphi(\boldx)$. Kernels make it possible to implicitly express
  large or even infinite dimensional basis features. We do this 
  by computing $\bphi(\boldx)^\top\bphi(\boldx')$ directly, without ever computing $\bphi(\boldx)$ .

  When training SVMs, we begin by computing the kernel matrix $\boldK$,
  over our training data $\{\boldx_1, \ldots, \boldx_n\}$.  The kernel
  matrix, defined as $K_{i, i'} = K(\boldx_i, \boldx_{i'})$, expresses
  the kernel function applied between all pairs of training points.

  In class, we saw Mercer's theorem, which tells us that any function
  $K$ that yields a positive semi-definite kernel matrix forms a valid
  kernel, i.e. corresponds to a matrix of dot-products under
  \textit{some} basis $\bphi$. Therefore instead of using an explicit
  basis, we can build kernel functions directly that fulfill this
  property.

  A particularly nice benefit of this theorem is that it allows us to
  build more expressive kernels by composition.  In this problem, you
  are tasked with using Mercer's theorem and the definition of a
  kernel matrix to prove that the following  compositions are valid kernels, 
  assuming $K^{(1)}$ and $K^{(2)}$ are valid kernels. Recall that a positive semi-definite matrix $\boldK$ requires $\mathbf{z}^\top \mathbf{Kz} \geq 0,\ \forall\ \mathbf{z} \in \reals^n$.

  \begin{enumerate}
  \item $K(\boldx, \boldx') = c\,K^{(1)}(\boldx, \boldx') \quad \text{for $c>0$}$
  \item $ 	K(\boldx, \boldx')= K^{(1)}(\boldx, \boldx') + K^{(2)}(\boldx, \boldx')$
  \item   $ K(\boldx, \boldx') = f(\boldx)\,K^{(1)}(\boldx, \boldx')\,f(\boldx') \quad
  \text{where $f$ is any function from~$\reals^m$ to $\reals$}$
  \item $ K(\boldx, \boldx') = K^{(1)}(\boldx, \boldx')\,K^{(2)}(\boldx,
  \boldx')$

  [Hint: Use the property that for any
  $\bphi(\boldx)$,
  $K(\boldx, \boldx') = \bphi(\boldx)^\top\bphi(\boldx')$ forms a
  positive semi-definite kernel matrix. ]
  \item 
  \begin{enumerate}
  	\item The $\exp$ function can be written as,
  	$$\exp(x) = \lim_{i\rightarrow \infty} \left(1 + x + \cdots + \frac{x^i}{i!}\right).$$
  	  Use this to show that $\exp(xx')$ (here
          $x, x'\in \reals$)) can be written as $\bphi(x)^\top \bphi(x')$ for some basis function $\bphi(x)$. Derive this basis function,
          and explain why this  would be hard to use as a basis in standard logistic regression.
  	\item Using the previous identities, show that $K(\boldx, \boldx') = \exp( K^{(1)}(\boldx, \boldx'))$ is a valid kernel.
  	

  \end{enumerate}
  \item  Finally use this analysis and previous identities to prove the validity of the Gaussian kernel:
  \begin{align*}
	K(\boldx, \boldx') &= \exp \left( \frac{-||\boldx - \boldx'||^2_2}{2\sigma^2} \right) 
  \end{align*}
  \end{enumerate}



 \end{problem}
\subsection*{Solution}


\newpage
%%%%%%%%%%%%%%%%%%%%%%%%%%%%%%%%%%%%%%%%%%%%%
% Problem 3
%%%%%%%%%%%%%%%%%%%%%%%%%%%%%%%%%%%%%%%%%%%%%
\begin{problem}[Scaling up your SVM solver, 10pts (+opportunity for extra credit)]


  For this problem you will build a simple SVM classifier for a binary
  classification problem. We have provided you two files for
  experimentation: training \textit{data.csv} and validation
  \textit{val.csv}.
\begin{itemize}
\item First read the paper at
  \url{http://www.jmlr.org/papers/volume6/bordes05a/bordes05a.pdf} and
  implement the Kernel Perceptron algorithm and the Budget Kernel
  Perceptron algorithm. Aim to make the optimization as fast as possible.
  Implement this algorithm in \textit{problem3.py}.

  [Hint: For this problem, efficiency will be an issue. Instead of directly
implementing this algorithm using numpy matrices, you should utilize
Python dictionaries to represent sparse matrices. This will be necessary 
to have the algorithm run in a reasonable amount of time.   
]
\item Next experiment with the hyperparameters for each of these
  models. Try seeing if you can identify some patterns by changing
  $\beta$, $N$ (the maximum number of support vectors), or the number
  of random training samples taken during the Randomized Search
  procedure (Section 4.3).  Note the training time, training and
  validation accuracy, and number of support vectors for various
  setups.
\item Lastly, compare the classification to the naive SVM imported from
scikit-learn by reporting accuracy on the provided validation
data. {\em For extra credit, implement the SMO algorithm and implement
  the LASVM process and do the same as above.}\footnote{Extra credit
  only makes a difference to your grade at the end of the semester if
  you are on a grade boundary.}

\end{itemize}


We are intentionally leaving this problem open-ended to allow for
experimentation, and so we will be looking for your thought process
and not a particular graph.  Visualizations should be generated 
using the provided code. You can use the trivial
$K(\boldx,\boldx') = \boldx^\top \boldx'$ kernel for this problem,
though you are welcome to experiment with more interesting kernels
too.


In addition, provide answers the following reading questions
{\bf in one or two sentences for each}.
%
\begin{enumerate}
\item In one short sentence, state the main purpose of the paper.
\item Describe each of the parameters in Eq.~1 in the paper
\item State, informally, one guarantee about the Kernel perceptron algorithm described in the
  paper. 
\item What is the main way the budget kernel perceptron algorithm tries to
  improve on the perceptron algorithm?
\item ({\em if you did the extra credit}) In simple words, what is the theoretical guarantee of LASVM algorithm? How
  does it compare to its practical performance?
\end{enumerate}


\end{problem}

\subsection*{Solution}



\newpage

\subsection*{Calibration [1pt]}
Approximately how long did this homework take you to complete?


\end{document}


















































